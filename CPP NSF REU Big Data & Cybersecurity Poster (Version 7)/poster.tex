\documentclass[12pt]{beamer}
\usepackage[orientation = portrait, size = custom, scale = 1.25, width=36in, height=48in]{beamerposter}
\usepackage[utf8]{inputenc}
\usepackage{amsmath}
\usepackage{amsfonts}
\usepackage{amssymb}
\usepackage{graphicx}
\usepackage{setspace}
\usepackage{xcolor}
\usepackage[useregional]{datetime2}
\usepackage{multicol}
\usepackage{caption}
\usepackage{subcaption}
\usepackage{tikz}
\documentclass[tikz,margin=0.5cm]{standalone}
\setbeamertemplate{caption}[numbered]
\usepackage[backend=biber, sorting=none]{biblatex}
\addbibresource{ref.bib}
\usepackage{filecontents}
\usepackage{svg}
\usepackage{graphicx}
\usepackage{array}
\usepackage{gensymb}
\usepackage{float}
\setlength{\columnsep}{80pt}
\usepackage{lipsum}
\usepackage{ragged2e}

\definecolor{azul}{rgb}{0.17, 0.40, 0.69}
\geometry{left = 2cm, right = 2cm, top = 2cm, bottom = 2cm}
\author{John Doe}
\institute{
	Prestigious University\\ \vspace{5mm}}

\title{Very Impressive Title}

\newcommand{\hacertitulo}{
		\begin{table}[h]
			\centering
			\begin{tabular}{ccc}
				\begin{minipage}{0.25\linewidth}
					\begin{figure}[h]
						\centering
						\includegraphics[height=12cm]{logo_sm1.jpg}
					\end{figure}
				\end{minipage} & \begin{minipage}{0.5\linewidth}
					\begin{center}
						\begin{spacing}{5}
							{\huge \textcolor{azul}{\textbf{\inserttitle}}}\\
						\end{spacing}
						\vspace{2cm}
						{\LARGE \insertauthor}\\
						\vspace{2cm}
						{\Large \textit{\insertinstitute}}
					\end{center}
				\end{minipage} & \begin{minipage}{0.25\linewidth}
					\begin{figure}[h]
						\centering
						\includegraphics[height=13cm]{}
					\end{figure}
				\end{minipage}
			\end{tabular}
		\end{table}
}


\newcommand{\seccion}[1]{
	\textcolor{azul}{\rule{\linewidth}{0.75mm}}\\
	\vspace{0.25cm}
	\begin{spacing}{3}
		\begin{center}
			{\large \textcolor{azul}{\textbf{#1}}}
		\end{center}
	\end{spacing}
	\textcolor{azul}{\rule{\linewidth}{0.75mm}}\\
	\vspace{0.25cm}
}

\begin{document}
	
	\hacertitulo{}
	
	
	\vspace{3.5cm}
	
	\justifying
	
	\begin{multicols}{2}
 
	\seccion{Introduction}
    \lipsum[1]
	\seccion{Background}
 \vspace{0.5cm}
    \lipsum[2]
        \seccion{Experiment}
        \lipsum[3]
        

\begin{figure}
    \centering
    \begin{tikzpicture}
   \newdimen\R
   \R=3cm
   \draw (0:\R) \foreach \x in {30,60,90,120,...,360} {  -- (\x:\R) };
   \foreach \x/\l/\p in
     {
     30/{$D$}/above,
     60/{$D^{\flat}$}/above,
     90/{$C$}/above,
      120/{$B$}/above,
      150/{$B^\flat$}/above,
      180/{$A$}/left,
      210/{$A^\flat$}/left,
      240/{$G$}/below,
      270/{$G^\flat$}/below,
      300/{$F$}/below,
      330/{$E$}/right,
      360/{$E^\flat$}/right
     }
     \node[inner sep=1pt,circle,draw,fill,label={\p:\l}] at (\x:\R) {};
    \end{tikzpicture}
    \caption{Dodecagon with Vertices labelled with corresponding Notes}
    \label{fig:musical-D_12}
\end{figure}
We can thus similarly represent the \textit{Circle of Fifths} as,
\begin{equation*}
    (r^7C)^n = \{C,G,D,A,E,B,G{^\flat},D{^\flat},A{^\flat},E{^\flat},B{^\flat},F\}.
\end{equation*}
	    \seccion{Musical Example}
\begin{figure}[H]
  \centering
    \includegraphics[scale=0.9]{HCB_orig.png}
\caption{First Two Measures of Hot Cross Buns}
    \label{fig:hot_cross_mes2}
\end{figure}

Here we take a look at the first two measures of the song \texit{Hot Cross Buns}. Let $H_{HCB}=\{B,A,G\}$ be the subset of $P_{12}$ which is the musical score for \textit{Hot Cross Buns} in $C$-major.
Then by the group action of $\mathbb{Z}_{12}$, for $h\in H_{HCB}$ and $z\in \Z_{12}$, we may musically transpose the score. Suppose $z = 7$, then for the pitch h = G in $H_{HCB}$, we see that
\begin{equation*}
\begin{split}
         7 + G &= \psi^{-1}((7 + \psi(G)) \bmod{12})\\
               &= \psi^{-1}((7 + 7) \bmod{12})\\
               &= D
\end{split}
\end{equation*}
so 7 + G = D under the $\mathbb{Z}_{12}$ group action, and so $7 + A = E$ and $7 +  = G{^\flat}$ similarly. Then we see the key has been musically transposed by a factor of 7, resulting in the following adjustment.

\begin{figure}[H]
  \centering
    \includegraphics[scale=3]{Z12transpose.JPG}
\caption{First Two Measures of Hot Cross Buns Musically Transposed to $G$-Major}
    \label{fig:hot_cross_mes2}
\end{figure}

Here we see that Hot Cross Buns has been musically transposed from the $C$-major to the $G$-major key.

Similarly, we can also achieve the musical transposition in Figure 4 of Hot Cross Buns by applying the group action of $D_{12}$ through the rotation $r^7$.


		\seccion{References}
		\printbibliography
	\end{multicols}
	\vspace{1cm}
	\noindent
	\textcolor{azul}{\rule{\linewidth}{0.75mm}}\\
	\vspace{1cm}
	\begin{center}
		{\Large \textcolor{azul}{\textbf{Conference Name - \monthdayyeardate\today}}}
	\end{center}
	

\end{document}